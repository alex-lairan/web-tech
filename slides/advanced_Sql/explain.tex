\begin{frame}
  \frametitle{Qu'est-ce que EXPLAIN en SQL?}

  \textbf{EXPLAIN} est une commande SQL qui permet de:
  \begin{itemize}
    \item Visualiser le plan d'exécution d'une requête.
    \item Comprendre comment le système de gestion de base de données (SGBD) va traiter votre requête.
    \item Identifier les goulots d'étranglement potentiels.
  \end{itemize}

  En d'autres termes, il aide à optimiser les performances des requêtes.
\end{frame}

\begin{frame}
  \frametitle{Pourquoi utiliser EXPLAIN?}

  Lorsque les requêtes deviennent complexes:
  \begin{itemize}
    \item Le SGBD choisit parmi plusieurs méthodes pour récupérer les données.
    \item Il peut utiliser des index, faire des scans complets, joindre les tables de différentes manières, etc.
    \item \textbf{EXPLAIN} montre le chemin choisi par le SGBD.
    \item Permet d'identifier si le SGBD utilise les bonnes ressources (par exemple, les index).
  \end{itemize}
\end{frame}

\begin{frame}[fragile]
  \frametitle{Comment utiliser EXPLAIN?}

  Très simple à utiliser :
  \begin{lstlisting}[language=SQL]
EXPLAIN SELECT * FROM table WHERE condition;
  \end{lstlisting}

  Le résultat vous donnera un aperçu du plan d'exécution, y compris :
  \begin{itemize}
    \item Type de scan (index ou full)
    \item Estimation du coût
    \item Tables et joints utilisés
    \item Et plus encore...
  \end{itemize}
\end{frame}

\begin{frame}
  \frametitle{Interpréter les résultats de EXPLAIN}

  \begin{itemize}
    \item `Seq Scan`: scan séquentiel de la table.
    \item `Index Scan`: utilisation d'un index.
    \item `Bitmap Index Scan`: création d'un bitmap à partir d'un index.
    \item Coût : estimé en unités de temps CPU et I/O.
    \item Et bien d'autres informations utiles pour l'optimisation.
  \end{itemize}

  Astuce : utiliser des outils graphiques ou des extensions pour une meilleure lisibilité.
\end{frame}

\begin{frame}
  \frametitle{Outils de Visualisation pour \texttt{EXPLAIN}}

  \begin{itemize}
    \item \textbf{PEV (Postgres EXPLAIN Visualizer)}
    \begin{itemize}
      \item Outil en ligne gratuit pour visualiser les plans \texttt{EXPLAIN} de PostgreSQL.
      \item \url{https://tatiyants.com/pev/}
    \end{itemize}

    \vspace{0.5em}

    \item \textbf{Depesz's Explain Analyzer}
    \begin{itemize}
      \item Analysez et visualisez les sorties de \texttt{EXPLAIN}.
      \item Annotations utiles pour une meilleure compréhension.
      \item \url{https://explain.depesz.com/}
    \end{itemize}

    \vspace{0.5em}

    \item \textbf{pgAdmin}
    \begin{itemize}
      \item Application de gestion pour PostgreSQL avec visualisation intégrée.
      \item Idéal pour ceux qui utilisent déjà pgAdmin pour la gestion de la base de données.
      \item \url{https://www.pgadmin.org/}
    \end{itemize}
  \end{itemize}
\end{frame}
