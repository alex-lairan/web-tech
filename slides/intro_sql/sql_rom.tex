\begin{frame}
  \frametitle{Introduction aux Relations avec ROM}
  Dans le cadre de ROM, les \textbf{relations} décrivent la structure des tables et leurs associations dans la base de données.

  \begin{itemize}
    \item Chaque relation correspond à une table dans la base de données.
    \item Les attributs de la relation correspondent aux colonnes de la table.
    \item Les associations définissent les liaisons entre les tables.
  \end{itemize}
\end{frame}

\begin{frame}[fragile]
  \frametitle{La Relation des Joueurs (`Players`)}

  \begin{lstlisting}[language=Ruby]
module Persistance
  module Relations
    class PlayersSql < ROM::Relation[:sql]
      schema(:players, as: :players_sql) do
        attribute :id, Types::PG::UUID
        attribute :name, Types::String
        primary_key :id
      end

      def for_ids(ids)
        where(id: ids)
      end
    end
  end
end
  \end{lstlisting}

  Cette relation décrit la table `players` avec ses attributs et une méthode supplémentaire pour filtrer par `ids`.
\end{frame}

\begin{frame}[fragile]
  \frametitle{La Relation des Jeux (`Games`)}

  \begin{lstlisting}[language=Ruby]
schema(:games, as: :games_sql) do
  attribute :id, Types::PG::UUID
  attribute :player_x_id, Types::PG::UUID
  attribute :player_o_id, Types::PG::UUID
  # ... (autres attributs)
  associations do
    has_many :moves
    belongs_to :players, as: :player_x, foreign_key: :player_x_id
    belongs_to :players, as: :player_o, foreign_key: :player_o_id
  end
end
  \end{lstlisting}

  La relation `Games` a des associations `belongs\_to` avec la table `Players`.
\end{frame}
