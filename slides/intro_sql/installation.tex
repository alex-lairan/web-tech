\begin{frame}[fragile]
  \frametitle{Installation d'un SGBD avec Docker}

  Docker facilite l'installation et l'exécution de SGBD de manière isolée. Voici comment installer PostgreSQL avec Docker :

  \begin{enumerate}
      \item \textbf{Récupérer l'image}:
      \begin{semiverbatim}
docker pull postgres:latest
      \end{semiverbatim}

      \item \textbf{Créer un conteneur}:
      \begin{semiverbatim}
docker run --name mon_postgres
  -e POSTGRES_PASSWORD=monpassword
  -p 5432:5432 -d postgres
      \end{semiverbatim}

      \item \textbf{Connexion}:
      \begin{itemize}
          \item Utilisez un client PostgreSQL pour vous connecter à `localhost` sur le port `5432` avec le mot de passe `monpassword`.
          \item Ou utilisez la CLI Docker :
          \begin{semiverbatim}
docker exec -it mon_postgres psql -U postgres
          \end{semiverbatim}
      \end{itemize}
  \end{enumerate}
\end{frame}

\begin{frame}[fragile]
  \frametitle{Utilisation de Docker Compose pour PostgreSQL}

  Docker Compose permet de définir et d'exécuter des applications Docker multi-conteneurs.
  \scriptsize
  \begin{semiverbatim}
version: '3'

services:
  db:
    image: postgres:14.1-alpine
    restart: always
    environment:
      - POSTGRES_USER=postgres
      - POSTGRES_PASSWORD=postgres
      - POSTGRES_DB=my_database
    ports:
      - '5432:5432'
    volumes:
      - db:/var/lib/postgresql/data
volumes:
  db:
    driver: local
  \end{semiverbatim}
  \normalsize
  Avec cette configuration, vous pouvez démarrer le service PostgreSQL simplement en exécutant : \texttt{docker-compose up}

\end{frame}
