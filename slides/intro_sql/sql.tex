\begin{frame}[fragile]
  \frametitle{Création de Tables en SQL}

  En SQL, la commande \texttt{CREATE TABLE} permet de créer une nouvelle table dans la base de données.

  \begin{semiverbatim}
CREATE TABLE nom\_de\_la\_table (
    colonne1 type\_de\_donnée1 contraintes1,
    colonne2 type\_de\_donnée2 contraintes2,
    ...
);
  \end{semiverbatim}

  \textbf{Exemple:}
  \begin{semiverbatim}
CREATE TABLE users (
    user\_id SERIAL PRIMARY KEY,
    username VARCHAR(100) NOT NULL,
    email VARCHAR(255) UNIQUE NOT NULL,
    password VARCHAR(255) NOT NULL,
    created\_at TIMESTAMP DEFAULT CURRENT\_TIMESTAMP
);
  \end{semiverbatim}

  Ici, \texttt{users} est une table avec plusieurs colonnes et contraintes associées.

\end{frame}

\begin{frame}
  \frametitle{Types de Données en SQL}

  Les types de données définissent le genre d'information que vous pouvez stocker dans une colonne spécifique d'une table. Voici quelques types courants:

  \begin{columns}
    \column{0.5\textwidth}
    \begin{itemize}
      \item \textbf{INTEGER}: Nombre entier
      \item \textbf{REAL, FLOAT}: Nombres à virgule flottante
      \item \textbf{DECIMAL(p, s)}: Nombres décimaux avec précision
      \item \textbf{CHAR(n)}: Chaîne de caractères de longueur fixe
      \item \textbf{VARCHAR(n)}: Chaîne de caractères de longueur variable
    \end{itemize}

    \column{0.5\textwidth}
    \begin{itemize}
      \item \textbf{DATE}: Date (année, mois, jour)
      \item \textbf{TIME}: Temps (heure, minute, seconde)
      \item \textbf{TIMESTAMP}: Date et heure
      \item \textbf{BOOLEAN}: Vrai ou Faux
      \item \textbf{BLOB}: Objet binaire (comme une image ou un fichier)
    \end{itemize}
  \end{columns}

  \vspace{0.25cm}
  \small
  Chaque Système de Gestion de Base de Données (SGBD) peut avoir ses propres types de données spécifiques, donc il est important de consulter la documentation du SGBD que vous utilisez.

\end{frame}

\begin{frame}
  \frametitle{Types de Données Spéciaux de PostgreSQL}

  PostgreSQL offre une variété de types de données uniques qui le distinguent des autres SGBD. Voici quelques-uns de ces types :

  \begin{columns}
    \column{0.5\textwidth}
    \begin{itemize}
      \item \textbf{SERIAL}: Nombre entier auto-incrémenté
      \item \textbf{UUID}: Identifiant universel unique
      \item \textbf{ARRAY}: Tableau d'éléments d'un type spécifié
      \item \textbf{JSON, JSONB}: Pour stocker des données au format JSON
      \item \textbf{HSTORE}: Paire clé-valeur
    \end{itemize}

    \column{0.5\textwidth}
    \begin{itemize}
      \item \textbf{INET}: Adresse IP
      \item \textbf{CIDR}: Bloc d'adresses IP
      \item \textbf{MACADDR}: Adresse MAC
      \item \textbf{TSVECTOR, TSQUERY}: Pour la recherche en texte intégral
      \item \textbf{ENUM}: Type énuméré
    \end{itemize}
  \end{columns}

  \vspace{0.5cm}
  Ces types spécifiques offrent plus de flexibilité et de fonctionnalités, optimisant ainsi la performance et la facilité d'utilisation de PostgreSQL.

\end{frame}

\begin{frame}
  \frametitle{PostGIS et ses Types de Données}

  \textbf{PostGIS} est une extension de PostgreSQL qui ajoute le support des types géographiques, permettant la réalisation de requêtes spatiales dans une base de données PostgreSQL.

  \begin{itemize}
    \item \textbf{Point} : Un point dans l'espace.
    \item \textbf{LineString} : Une ligne composée de deux points ou plus.
    \item \textbf{Polygon} : Une forme à deux dimensions.
    \item \textbf{MultiPoint} : Plusieurs points.
    \item \textbf{MultiLineString} : Plusieurs lignes.
    \item \textbf{MultiPolygon} : Plusieurs formes.
    \item \textbf{GeometryCollection} : Une collection de formes.
    \item \textbf{BOX2D \& BOX3D} : Bounding box 2D et 3D.
  \end{itemize}

  Outre ces types, PostGIS fournit de nombreuses fonctions pour créer, manipuler et interroger des données spatiales.

  \vspace{0.25cm}
  Avec PostGIS, PostgreSQL peut rivaliser avec d'autres SGBD spatiaux dédiés en termes de fonctionnalités et de performance.

\end{frame}


\begin{frame}
  \frametitle{Supprimer une Table : DROP TABLE}

  La commande \textbf{DROP TABLE} est utilisée pour supprimer une table existante dans une base de données. Elle supprime la table et toutes ses données de façon permanente.

  \begin{itemize}
    \item \textbf{Syntaxe}:
      \begin{semiverbatim}
DROP TABLE nom\_de\_la\_table;
      \end{semiverbatim}

    \item \textbf{Points à considérer}:
      \begin{itemize}
        \item La suppression est \alert{irréversible}. Une fois que vous exécutez cette commande, les données sont perdues.
        \item Assurez-vous d'avoir une sauvegarde des données avant d'utiliser cette commande si nécessaire.
        \item Soyez prudent lorsque vous utilisez cette commande dans un environnement de production.
      \end{itemize}

  \end{itemize}

  \textbf{Exemple} :
      \begin{semiverbatim}
DROP TABLE employes;
      \end{semiverbatim}
  Ceci supprimera la table "employes" et toutes ses données.

\end{frame}


\begin{frame}[fragile]
  \frametitle{Modification de Tables en SQL}

  Pour ajouter un champ à une table existante, on utilise la commande \texttt{ALTER TABLE} suivie de \texttt{ADD COLUMN}.

  \begin{semiverbatim}
ALTER TABLE nom\_de\_la\_table
ADD COLUMN nouveau\_champ type\_de\_donnée contraintes;
  \end{semiverbatim}

  \textbf{Exemple:}
  \begin{semiverbatim}
ALTER TABLE users
ADD COLUMN last\_login TIMESTAMP;
  \end{semiverbatim}

  Avec cette commande, un nouveau champ \texttt{last\_login} de type \texttt{TIMESTAMP} est ajouté à la table \texttt{users}.

\end{frame}

\begin{frame}[fragile]
  \frametitle{Retrait d'un champ d'une table en SQL}

  Pour retirer un champ d'une table existante, on utilise la commande \texttt{ALTER TABLE} suivie de \texttt{DROP COLUMN}.

  \begin{semiverbatim}
ALTER TABLE nom\_de\_la\_table
DROP COLUMN nom\_du\_champ;
  \end{semiverbatim}

  \textbf{Exemple:}
  \begin{semiverbatim}
ALTER TABLE users
DROP COLUMN last\_login;
  \end{semiverbatim}

  Après l'exécution de cette commande, le champ \texttt{last\_login} sera supprimé de la table \texttt{users}.

\end{frame}

\begin{frame}[fragile]
  \frametitle{Clés Étrangères en SQL}

  Une \textbf{clé étrangère} est une colonne ou un ensemble de colonnes dans une table qui fait référence à la clé primaire d'une autre table.

  \begin{itemize}
    \item Garantit l'\textit{intégrité référentielle} des données.
    \item Permet de créer des relations entre les tables.
  \end{itemize}

  \begin{semiverbatim}
ALTER TABLE nom\_de\_la\_table\_enfant
ADD FOREIGN KEY (nom\_du\_champ)
REFERENCES nom\_de\_la\_table\_parent(clé\_primaire);
  \end{semiverbatim}

  \textbf{Exemple:}
  \begin{semiverbatim}
ALTER TABLE orders
ADD FOREIGN KEY (user\_id)
REFERENCES users(user\_id);
  \end{semiverbatim}

  Ici, \texttt{user\_id} dans la table \texttt{orders} fait référence à \texttt{user\_id} dans la table \texttt{users}.

\end{frame}

\begin{frame}[fragile]
  \frametitle{La commande SELECT en SQL}

  La commande \texttt{SELECT} permet d'interroger une ou plusieurs tables pour récupérer des données.

  \begin{semiverbatim}
SELECT colonnes
FROM nom\_de\_la\_table
WHERE condition;
  \end{semiverbatim}

  \begin{itemize}
    \item \textbf{colonnes}: Spécifiez les colonnes que vous souhaitez récupérer. Utilisez `*` pour toutes les colonnes.
    \item \textbf{nom\_de\_la\_table}: La table à partir de laquelle vous souhaitez récupérer les données.
    \item \textbf{WHERE}: Permet de filtrer les résultats selon une condition spécifiée.
  \end{itemize}

  \textbf{Exemple:}
  \begin{semiverbatim}
SELECT username, email
FROM users
WHERE user\_id > 100;
  \end{semiverbatim}

\end{frame}


\begin{frame}[fragile]
  \frametitle{Les Jointures en SQL}

  Les jointures permettent de \textbf{combiner} des lignes de deux ou plusieurs tables en fonction d'une colonne liée entre elles.

  \begin{itemize}
    \item \textbf{INNER JOIN}: Renvoie les lignes quand il y a au moins une correspondance dans les deux tables.
    \item \textbf{LEFT (OUTER) JOIN}: Renvoie toutes les lignes de la table de gauche, et les correspondances de la table de droite.
    \item \textbf{RIGHT (OUTER) JOIN}: Renvoie toutes les lignes de la table de droite, et les correspondances de la table de gauche.
    \item \textbf{FULL (OUTER) JOIN}: Renvoie les lignes quand il y a une correspondance dans l'une des tables.
  \end{itemize}

  \textbf{Exemple avec INNER JOIN:}
  \begin{semiverbatim}
SELECT users.username, orders.order\_id
FROM users
INNER JOIN orders ON users.user\_id = orders.user\_id;
  \end{semiverbatim}

  Cette requête renvoie tous les usernames et les order\_ids où il existe une correspondance entre les user\_ids des deux tables.

\end{frame}

\begin{frame}[fragile]
  \frametitle{La commande INSERT INTO en SQL}

  La commande \texttt{INSERT INTO} permet d'ajouter de nouvelles lignes à une table.

  \begin{semiverbatim}
INSERT INTO nom\_de\_la\_table (colonne1, colonne2, ...)
VALUES (valeur1, valeur2, ...);
  \end{semiverbatim}

  \begin{itemize}
    \item \textbf{nom\_de\_la\_table}: La table dans laquelle vous souhaitez insérer des données.
    \item \textbf{colonne1, colonne2, ...}: Les colonnes pour lesquelles vous spécifiez des valeurs.
    \item \textbf{valeur1, valeur2, ...}: Les valeurs à insérer pour les colonnes spécifiées.
  \end{itemize}

  \textbf{Exemple:}
  \begin{semiverbatim}
INSERT INTO users (username, email)
VALUES ('jdoe', 'jdoe@example.com');
  \end{semiverbatim}

\end{frame}

\begin{frame}[fragile]
  \frametitle{La commande UPDATE en SQL}

  La commande \texttt{UPDATE} permet de modifier des données existantes dans une table.

  \begin{semiverbatim}
UPDATE nom\_de\_la\_table
SET colonne1 = valeur1, colonne2 = valeur2, ...
WHERE condition;
  \end{semiverbatim}

  \begin{itemize}
    \item \textbf{nom\_de\_la\_table}: La table dans laquelle vous souhaitez modifier des données.
    \item \textbf{colonne1 = valeur1, ...}: Paire colonne/valeur spécifiant les nouvelles données.
    \item \textbf{WHERE}: La condition qui détermine quelles lignes doivent être mises à jour.
  \end{itemize}

  \textbf{Exemple:}
  \begin{semiverbatim}
UPDATE users
SET email = 'john.doe@example.com'
WHERE username = 'jdoe';
  \end{semiverbatim}

\end{frame}

\begin{frame}[fragile]
  \frametitle{La commande DELETE en SQL}

  La commande \texttt{DELETE} permet de supprimer des enregistrements d'une table en fonction d'une condition donnée.

  \begin{semiverbatim}
DELETE FROM nom\_de\_la\_table
WHERE condition;
  \end{semiverbatim}

  \begin{itemize}
    \item \textbf{nom\_de\_la\_table}: La table de laquelle vous souhaitez supprimer des enregistrements.
    \item \textbf{WHERE}: La condition qui détermine quelles lignes doivent être supprimées.
  \end{itemize}

  \textbf{Exemple:}
  \begin{semiverbatim}
DELETE FROM users
WHERE username = 'jdoe';
  \end{semiverbatim}

  Cette requête supprime l'utilisateur avec le nom d'utilisateur `jdoe`.

  \alert{Attention!} Sans la clause \texttt{WHERE}, vous supprimerez \textbf{tous} les enregistrements de la table!

\end{frame}

\begin{frame}
  \frametitle{Les Index en SQL}

  Un \textbf{index} est une structure de données qui améliore la vitesse des opérations dans une base de données. Il permet à la base de données de trouver rapidement des enregistrements sans avoir à parcourir chaque enregistrement de la table chaque fois qu'une opération de base de données est effectuée.

  \begin{itemize}
    \item \textbf{Avantages}:
      \begin{itemize}
        \item Augmente la vitesse des opérations de recherche (SELECT).
        \item Accélère les opérations de jointure entre tables.
      \end{itemize}
    \item \textbf{Inconvénients}:
      \begin{itemize}
        \item Prend de l'espace de stockage supplémentaire.
        \item Peut ralentir les opérations d'écriture (INSERT, UPDATE, DELETE) car l'index doit être mis à jour.
      \end{itemize}
    \item \textbf{Création d'un index}:
      \begin{semiverbatim}
CREATE INDEX nom\_de\_l\_index
ON nom\_de\_la\_table (colonne);
      \end{semiverbatim}
    \item \textbf{Suppression d'un index}:
      \begin{semiverbatim}
DROP INDEX nom\_de\_l\_index;
      \end{semiverbatim}
  \end{itemize}

  \textbf{Conseil} : N'indexez pas chaque colonne. Indexez les colonnes qui sont fréquemment recherchées ou jointes.

\end{frame}

\begin{frame}
  \frametitle{Les Migrations de Bases de Données}

  Une \textbf{migration} fait référence au processus de modification de la structure ou des données d'une base de données entre deux versions d'une application.

  \begin{itemize}
    \item \textbf{Pourquoi les migrations ?}
      \begin{itemize}
        \item Adapter la base de données à de nouvelles exigences.
        \item Corriger des erreurs de conception précédentes.
        \item Intégrer de nouvelles fonctionnalités sans perturber les données existantes.
      \end{itemize}

    \item \textbf{Outils populaires}:
      \begin{itemize}
        \item \textbf{Liquibase}
        \item \textbf{Flyway}
        \item \textbf{Django Migrations} (pour les utilisateurs de Django)
        \item \textbf{Alembic} (souvent utilisé avec SQLAlchemy)
      \end{itemize}

  \end{itemize}

\end{frame}

\begin{frame}
  \frametitle{Les Migrations de Bases de Données}
  \begin{itemize}
    \item \textbf{Processus typique}:
      \begin{enumerate}
        \item Écrire un script de migration pour décrire les changements.
        \item Tester la migration dans un environnement de développement.
        \item Appliquer la migration dans l'environnement de production.
        \item Versionner les migrations pour garder une trace des modifications.
      \end{enumerate}

  \end{itemize}

  \textbf{Note}: La gestion des migrations est essentielle pour assurer l'intégrité et la consistance des données lors de l'évolution d'une application.

\end{frame}
